\documentclass[a4paper]{article}

%% Language and font encodings
\usepackage[english]{babel}
\usepackage[utf8x]{inputenc}
\usepackage[T1]{fontenc}

%% Sets page size and margins
\usepackage[a4paper,top=3cm,bottom=2cm,left=3cm,right=3cm,marginparwidth=1.75cm]{geometry}

%% Useful packages
\usepackage{amsmath}
\usepackage{amsfonts}
\usepackage{graphicx}
\usepackage[colorinlistoftodos]{todonotes}
\usepackage[colorlinks=true, allcolors=blue]{hyperref}

\title{Judson's Abstract Algebra: Chapter 9}
\date{}

\begin{document}
\maketitle

\section*{1}

Prove that $\mathbb{Z} \cong n \mathbb{Z}$ for $n \neq 0$.

\vspace{\baselineskip}

Let $n \in \mathbb{Z}, n \neq 0$. Consider the function $f : \mathbb{Z} \rightarrow n \mathbb{Z}$

$$f(x) = nx.$$

Let $x,y \in \mathbb{Z}$. Note that 

$$f(x+y) = n(x+y) = nx + ny = f(x) + f(y).$$

Assume $f(x) = f(y)$. Then

$$f(x) = f(y)$$
$$nx = ny$$
$$x = y$$

so $f$ is injective. Note that $f$ is surjective by definition. Since $f$ is a bijection and preserves the group operation $f$ is an isomorphism.


\section*{2} 

Prove that $\mathbb{C}^*$ is isomorphic to the subgroup of $GL_2(\mathbb{R}$ consisting of matrices of form

$$
  \begin{pmatrix}
    a & b \\
    -b & a
  \end{pmatrix}.$$
  
We will call the subgroup described above $M$. Consider a function $f : \mathbb{C}^* \rightarrow M$ defined by 

$$f(a+bi) = \begin{pmatrix}
    a & b \\
    -b & a
  \end{pmatrix}.$$
  
Let $a,b,c,d \in \mathbb{R}$ where $ab \neq 0$ and $cd \neq 0$. First note that the function preserves the group operation since

\begin{align*}
f((a+bi)(c+di)) &= f(ac-db + (db + da)i) \\
&= \begin{pmatrix}
    ac-db & cb + da \\
    cb + da & ac - db
  \end{pmatrix} \\
&= = \begin{pmatrix}
    a & b \\
    -b & a
  \end{pmatrix}
  = \begin{pmatrix}
    c & d \\
    -d & c
  \end{pmatrix} \\
&= f(a+bi)f(c+di)
\end{align*}

Note that $f$ is surjective because $a,b$ are arbitrary real numbers. Notice that $f$ is injective because of element-wise equality of matrices.


\section*{7}

Show that any cyclic group of order $n$ is isomorphic to $\mathbb{Z}_n$.

\vspace{\baselineskip}

Let $G$ be a cyclic group of order $n$ generated by $a$ and define a function $\phi : \mathbb{Z}_n \rightarrow G$ by $\phi(k) = a^k$ where $0 \leq k < n$. Let $x,y \in \mathbb{Z}_n$. 
Notice that since

$$\phi(x + y) = a^{x+y} = a^x a^y = \phi(x) \phi(y)$$

that $\phi$ preserves the group operation. Assume $\phi(x) = \phi(y)$. Then

$$\phi(x) = \phi(y)$$
$$a^x = a^y$$
$$a^x a^{-y} = e$$
$$a^{x-y} = e.$$

Hence $x - y \equiv 0 \mod n$ and therefore $x \equiv y \mod n$ and $\phi$ is injective. Let $g \in G$. Since $G$ is cyclic, there exists $k \in \mathbb{Z}, 0 \leq k < n$ such that

$$g = a^k = \phi(k)$$

therefore $\phi$ is surjective. This proves that $\phi$ is an isomorphism, hence any cyclic group of order $n$ is isomorphic to $\mathbb{Z}_n$.


\section*{27}

Let $G \equiv H$. Show that if $G$ is cyclic, then so is $H$.

\vspace{\baselineskip}

Let $a$ be a generator of $G$. Then for all $g \in G$ there exists $k \in \mathbb{Z}$ such that $a^k = g$. Let $f: G \rightarrow H$ be an isomorphism. Let $h \in H$. There exists $g \in G$ such that

\begin{align*}
h &= f(g) \\
&= f(a^k) \\
&= f(a)^k.
\end{align*}

Notice that $f(a)$ is a generator of $H$ since $h$ is arbitrary. This proves that $H$ is a cyclic group.


\section*{34}

An \textit{automorphism} of a group $G$ is an isomorphism with itself. Prove that complex conjugation is an automorphism of the additive group of complex numbers; that is, show that the map $\phi(a+bi) = a - bi$ is an isomorphism from $\mathbb{C}$ to $\mathbb{C}$.

\vspace{\baselineskip}

Note that since

\begin{align*}
\phi((a+bi) + (c+di)) &= \phi((a+c) + (b+d)i) \\
&= (a+c) - (b+d)i \\
&= (a-bi) + (c-di) \\
&= \phi(a+bi) + \phi(c+di)
\end{align*}

we know that $\phi$ preserves the group operation. The proof that $\phi$ is surjective is trivial (to get any element apply $\phi$ to the element's inverse). To show that $\phi$ is injective consider

$$\phi(a+bi) = \phi(c+di)$$
$$\overline{\phi(a+bi)} = \overline{\phi(c+di)}$$
$$a + bi = c + di.$$




\end{document}