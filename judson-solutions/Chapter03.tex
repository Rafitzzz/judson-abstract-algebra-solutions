\documentclass[a4paper]{article}

%% Language and font encodings
\usepackage[english]{babel}
\usepackage[utf8x]{inputenc}
\usepackage[T1]{fontenc}

%% Sets page size and margins
\usepackage[a4paper,top=3cm,bottom=2cm,left=3cm,right=3cm,marginparwidth=1.75cm]{geometry}

%% Useful packages
\usepackage{amsmath}
\usepackage{amsfonts}
\usepackage{graphicx}
\usepackage[colorinlistoftodos]{todonotes}
\usepackage[colorlinks=true, allcolors=blue]{hyperref}

\title{Judson's Abstract Algebra: Chapter 3}
\date{}

\begin{document}
\maketitle

\section{}

Suppose $3x \equiv 2 \mod 7$. Then $x \in \{z: z = 3 + 7n, n \in \mathbb{Z}\}$.

\vspace{\baselineskip}

Suppose $5x + 1 \equiv 13 \mod 23$. Then $x \in \{z: z = 7 + 23n, n \in \mathbb{Z}\}$.

\vspace{\baselineskip}

Suppose $5x + 1 \equiv 13 \mod 26$. Then $x \in \{z: z = 18 + 26n, n \in \mathbb{Z}\}$.

\vspace{\baselineskip}

Suppose $9x \equiv 3 \mod 5$. Then $x \in \{z: z = 2 + 5n, n \in \mathbb{Z}\}$.

\vspace{\baselineskip}

Suppose $5x \equiv 1 \mod 6$. Then $x \in \{z: z = 5 + 6n, n \in \mathbb{Z}\}$.

\vspace{\baselineskip}

Suppose $3x \equiv 1 \mod 6$. There are no solutions.

\section{}

\section{}

\section{}

\section{}

\section{}

\section{}

Let $S = \mathbb{R} \setminus \{ -1 \}$ and define a binary operation on $S$ by $a * b = a + b + ab$. Prove that $(S, *)$ is an abelian group.

\vspace{\baselineskip}

Let $a, b, c \in S$. The identity is 0 since $a * 0 = a + 0 + 0a = a$. The inverse of $a$ is given by 

$$ a^{-1} = \frac{-a}{1+a}$$

since

\begin{align*}
a * a^{-1} &= a * \frac{-a}{1+a} \\
&= a + \frac{-a}{1+a} + \frac{-a^2}{1+a} \\
&= \frac{a + a^2}{1+a} - \frac{a + a^2}{1+a} \\
&= 0
\end{align*}

and $a \neq -1$.

Consider

\begin{align*}
(a * b) * c &= (a + b + ab) * c \\
&= a + b + ab + c + c (a + b + ab) \\
&= a + b + c + ab + ac + bc + abc \\
&= a + (b + c + bc) + (ab + ac + abc) \\
&= a + (b * c) + a (b * c) \\
&= a * (b * c).
\end{align*}

This shows associativity.

\section{}

Give an example of two elements $A$ and $B$ in $GL_2(\mathbb{R})$ with $AB \neq BA$.

\vspace{\baselineskip}

Consider

$$A=
  \begin{pmatrix}
    1 & 1 \\
    0 & 1
  \end{pmatrix}
  B=
  \begin{pmatrix}
    1 & 0 \\
    1 & 1
  \end{pmatrix}.
$$

Notice

$$AB =   
  \begin{pmatrix}
    2 & 1 \\
    1 & 1
  \end{pmatrix}
$$

while 

$$BA = 
  \begin{pmatrix}
    1 & 1 \\
    1 & 2
  \end{pmatrix}
$$

\section{}

Prove that the product of two matrices in $SL_2(\mathbb{R})$ has determinant one.

\vspace{\baselineskip}

Let $A,B \in SL_2(\mathbb{R})$. This follows from the basic property of determinants

$$\det (AB) = \det(A) \det(B) = 1 \cdot 1 = 1.$$

\section{}

Prove that the set of matrices of the form
$$
  \begin{pmatrix}
    1 & x & y \\
    0 & 1 & z \\
    0 & 0 & 1
  \end{pmatrix}
$$

is a group under matrix multiplication. This group, known as the \textit{Heisenburg group}, is important in quantum physics. Matrix multiplication in the Heisenberg group is defined by

$$
  \begin{pmatrix}
    1 & x & y \\
    0 & 1 & z \\
    0 & 0 & 1
  \end{pmatrix}
  \begin{pmatrix}
    1 & x' & y' \\
    0 & 1 & z' \\
    0 & 0 & 1
  \end{pmatrix}
=
  \begin{pmatrix}
    1 & x + x' & y + y'+ xz' \\
    0 & 1 & z + z' \\
    0 & 0 & 1
  \end{pmatrix}
$$

The proof that this set of matrices forms a group is straightforward. Associativity follows from the associativity of matrix multiplication. The identity element is the usual identity matrix. Each matrix in the set is upper triangular and has an inverse (if this is not obvious, then the matrices can be easily put in reduced row echelon form).


\section{}

Prove that $\det(AB) = det(A)det(B)$ in $GL_2(\mathbb{R})$ Use this result to show that the binary operation in the group $GL_2(\mathbb{R})$ is closed; that is, if $A$ and $B$ are in $GL_2(\mathbb{R})$, then $AB \in GL_2(\mathbb{R})$.

\vspace{\baselineskip}

We already used this basic property of determinants in problem 9. We will prove it for $2 \times 2$ matrices here. Recall the definition of the determinant for a $2 \times 2$ matrix 
$$A = 
  \begin{pmatrix}
    a & b \\
    c & d \\
  \end{pmatrix}
$$
  
is given by $\det(A) = ad - bc$.

\vspace{\baselineskip}

Let $A,B$ be $2 \times 2$ matrices where  

$$A = 
  \begin{pmatrix}
    a & b \\
    c & d \\
  \end{pmatrix}
  B = 
  \begin{pmatrix}
    a' & b' \\
    c' & d' \\
  \end{pmatrix}
$$

and 

$$AB = 
  \begin{pmatrix}
    aa' + bc' & ab' + bd'\\
    a'c + c'd & b'c + dd' \\
  \end{pmatrix}.
$$

Notice that 

\begin{align*}
\det(AB) &= (aa' + bc')(b'c + dd') - (ab' + bd')(a'c + c'd) \\
&= aa'b'c + aa'dd' + bc'b'c + bc'dd' - ab'a'c - ab'c'd - bd'a'c - bd'c'd \\
&= (ad - bc) (a'd' - b'c') \\
&= \det(A) \det(B)
\end{align*}

\vspace{\baselineskip}

Let $C,D \in GL_2(\mathbb{R})$. Then $\det(C) \neq 0$ and $\det(D) \neq 0$. By the property proved above $\det{CD} = \det{C}\det{D}$. Since $\det{C} \neq 0$ and $\det{D} \neq 0$ we know that $\det{CD} \neq{0}$ hence $CD \in GL_2(\mathbb{R})$ and $GL_2(\mathbb{R})$ is closed.


\section{}



\section{}

Show that $\mathbb{R}^{*} = \mathbb{R} \setminus \{ 0 \}$ is a group under the operation of multiplication.

\vspace{\baselineskip}

Let $x \in \mathbb{R}^{*}$. Notice that, since multiplication of two non-zero real numbers is non-zero,$\mathbb{R}^{*}$ is closed. Associativity follows from the associativity of the multiplication of real numbers. The identity is 1. Notice that $x^{-1} = \frac{1}{x}$ since

$$xx^{-1} = x \frac{1}{x} = 1.$$

Hence, $\mathbb{R}^*$ is a group


\section{}

Given the groups $\mathbb{R}^*$ and $\mathbb{Z}$ let $G = \mathbb{R} \times \mathbb{Z}$. Define a binary operation $\circ$ on $G$ by $(a,m) \circ (b,n) = (ab, m+n)$. Show that $G$ is a group under this operation.

\vspace{\baselineskip}

This proof is straightforward since the two contributing groups are operated on independently. Associativity follows from the associativity of $\mathbb{R}^*$ and $\mathbb{Z}$. The identity is $(1,0)$. The inverse of $(x, n) \in \mathbb{R}^* \times \mathbb{Z}$ is $(1/x, -n)$.

\section{}

The dihedral group of six elements is the smallest non-abelian group. Here is the Cayley table

\begin{tabular}{ l | l l l l l l }
  * & e & a & b & c & d & f \\
  \hline      
  e & e & a & b & c & d & f \\
  a & a & e & d & f & b & c \\
  b & b & f & e & d & c & a \\
  c & c & d & f & e & a & b \\
  d & d & c & a & b & f & e \\
  f & f & b & c & a & e & d
\end{tabular}

\section{}


\section{}


\section{}


\section{}


\section{}


\section{}


\section{}



\section{}


\section{}

Let $a$ and $B$ be elements in a group $G$. Prove that $ab^na^{-1} = (aba^{-1})^n$ for $n \in \mathbb{Z}$.

\vspace{\baselineskip}

Consider

$$ab^na^{-1} = a b ... b  a^{-1} = a  b a^{-1} a ... a^{-1} a b a^{-1} = (aba^{-1})^n.$$


\section{}


\section{}

Prove that the inverse of $g1 g2 ... g_n$ is $g_n^{-1} g_{n-1}^{-1} ... g_1^{-1}$.

\vspace{\baselineskip}

Consider

\begin{align*}
(g1 g2 ... g_{n-1} g_n) (g_n^{-1} g_{n-1}^{-1} ... g_2^{-1} g_1^{-1}) &= g1 g2 ... g_{n-1} g_{n-1}^{-1} ... g_2^{-1} g_1^{-1} \\
\vdots \\
&= g_1 g_2 g_2^{-1} g_1^{-1} \\
&= g_1 g_1^{-1} \\
&= e.
\end{align*}


\section{}

Prove the remainder of Proposition 3.6: if $G$ is a group and $a,b \in G$, then the equation $xa = b$ has unique solutions in $G$.

\vspace{\baselineskip}

Assume that $xa = b$ has more than one solution. Call two solutions $x_1, x_2$. Note that 

\begin{align*}
x_1 a &= b \\
x_1 a a^{-1} &= b a^{-1} \\
x_1 &= b a^{-1}
\end{align*}

and 

\begin{align*}
x_2 a &= b \\
x_2 a a^{-1} &= b a^{-1} \\
x_2 &= b a^{-1}
\end{align*}

Then, since the group operation is well-defined $x_1 = x_2$, a contradiction. This proves that solutions to $xa = b$ in $G$ are unique.









\end{document}