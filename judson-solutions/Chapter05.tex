\documentclass[a4paper]{article}

%% Language and font encodings
\usepackage[english]{babel}
\usepackage[utf8x]{inputenc}
\usepackage[T1]{fontenc}

%% Sets page size and margins
\usepackage[a4paper,top=3cm,bottom=2cm,left=3cm,right=3cm,marginparwidth=1.75cm]{geometry}

%% Useful packages
\usepackage{amsmath}
\usepackage{amsfonts}
\usepackage{graphicx}
\usepackage[colorinlistoftodos]{todonotes}
\usepackage[colorlinks=true, allcolors=blue]{hyperref}

\title{Judson's Abstract Algebra: Chapter 5}
\date{}

\begin{document}
\maketitle

\section*{1}

Write the following permutations in cycle notation

$$
\begin{pmatrix}
1 & 2 & 3 & 4 & 5 \\
2 & 4 & 1 & 5 & 3
\end{pmatrix}
= (12453)$$

$$
\begin{pmatrix}
1 & 2 & 3 & 4 & 5 \\
4 & 2 & 5 & 1 & 3
\end{pmatrix}
= (14)(35)$$

$$
\begin{pmatrix}
1 & 2 & 3 & 4 & 5 \\
3 & 5 & 1 & 4 & 2
\end{pmatrix}
= (13)(25)$$

$$
\begin{pmatrix}
1 & 2 & 3 & 4 & 5 \\
1 & 4 & 3 & 2 & 5
\end{pmatrix}
= (24)$$


\section*{2}

Compute each of the following

$$(1345)(234) = (135)(24)$$
$$(12)(1253) = (253)$$
$$(143)(23)(24) = (14)(23)$$
$$(1423)(34)(56)(1324) = (12)(56)$$
$$(1254)(13)(25) = (1324)$$
$$(1254)(13)(25)^2 = (1254)(13) = (13254)$$
$$(1254)^{-1}(123)(45)(1254) = (1432)(123)(45)(1254) = (134)(25)$$
$$(1254)^2(123)(45) = (14)(235)$$
$$(123)(45)(1254)^{-2} = (123)(54)(1452)^2 = (143)(25)$$
$$(1254)^{100} = ((1254)^4)^{25} = (1)^{25} = (1)$$
$$|(1254)| = 4$$
$$|(1254)^2| = 2$$
$$(12)^{-1} = (12)$$
$$(12537)^{-1} = (17352)$$
$$[(12)(34)(12)(47)]^{-1} = [(34)(47)]^{-1} = (374)$$
$$[(1235)(467)]^{-1} = (1532)(476)$$


\section*{3}

Express the following permutations as products of transpositions and identify them as even or odd.

$$(14356) = (14)(43)(35)(56)$$
$$(156)(234) = (15)(56)(23)(34)$$
$$(1426)(142) = (1246) = (12)(24)(46)$$
$$(17254)(1423)(154632) = (14672) = (14)(46)(67)(72)$$
$$(142637) = (14)(42)(26)(63)(37)$$



\section*{4}

Find $(a_1, a_2, ..., a_n)^{-1}$.

$$(a_1, a_2, ..., a_n)^{-1} = (a_1, a_n, a_{n-1}, ..., a_2)$$


\section*{8}

Show that $A_8$ contains an element of order 15.

\vspace{\baselineskip}

Note that $(12345)(678)$ has order 15. To show that $(12345)(678)$ has even order, consider

$$(12345)(678) = (12)(23)(34)(45)(67)(78).$$


\section*{17}

Prove that $S_n$ is nonabelian for $n \geq 3$.

\vspace{\baselineskip}

Notice that the following are two unequal products in $S_n$ for all $n \geq 3$ 

$$(123)(12) = (13) \neq (23) = (12)(123).$$



\section*{24}

Show that a 3-cycle is an even permutation.

\vspace{\baselineskip}

Consider a 3-cycle $(a_1, a_2, a_3)$. Notice that 

$$(a_1, a_2, a_3) = (a_1, a_2)(a_2, a_3).$$

This proves that all 3-cycles are even permutations.


\section*{27}

Let $G$ be a group and define a map $\lambda_g : G \rightarrow G$ by $\lambda_g(a) = ga$. Prove that $\lambda_g$ is a permutation of $G$.

\vspace{\baselineskip}

We must show that $\lambda_g$ is a bijection. Let $x_1, x_2 \in G$. Since

\begin{align*}
\lambda_g(x_1) &= \lambda_g(x_2) \\
gx_1 &= gx_2 \\
g^{-1} g x_1 &= g^{-1} g x_2 \\
x_1 &= x_2
\end{align*}

it is true that $\lambda_g$ is injective. Notice that since

$$g (g^{-1} x_1) = x_1$$

it is true that $\lambda_g$ is surjective. 


\section*{28}

Prove that there exist $n!$ permutations of a set containing $n$ elements.

\vspace{\baselineskip}

Notice that the first element has $n$ possibilities, the second element  has $n-1$ possibilities ... the last element has $1$ possibility. This yields

$$n(n-1)...1 = n!.$$


\section*{34}

If $\alpha$ is even, prove that $\alpha^{-1}$ is even. Does a corresponding result hold if $\alpha$ is odd?

\vspace{\baselineskip}

First note that the identity permutation is even. Assume that $\alpha^{-1}$ is odd. Then note that $\alpha \alpha^{-1} = (1)$ implies that the identity is odd, a contradiction. The proof for the corresponding result when $\alpha$ is odd is identical.


\section*{35}

Show that $\alpha^{-1} \beta^{-1} \alpha \beta$ is even for $\alpha, \beta \in S_n$.

\vspace{\baselineskip}

First note that by problem 34 $x^{-1}$ is even if $x$ is even and odd if $x$ is odd for all $x \in S_n$. There are 4 (nearly identical cases). If $\alpha$ and $\beta$ are even then $\alpha^{-1}$ and $\beta^{-1}$ are even as well. Clearly $\alpha^{-1} \beta^{-1} \alpha \beta$ is even. If $\alpha$ and $\beta$ are odd then $\alpha^{-1}$ and $\beta^{-1}$ are odd as well. Since an odd plus an odd is an even, clearly $\alpha^{-1} \beta^{-1} \alpha \beta$ is even. The final two cases are the same. Without loss of generality assume $\alpha$ is even and $\beta$ is odd. Then $\alpha^{-1}$ is even and $\beta^{-1}$ is odd. Since an odd plus an odd is an even, clearly $\alpha^{-1} \beta^{-1} \alpha \beta$ is even.







\end{document}