\documentclass[a4paper]{article}

%% Language and font encodings
\usepackage[english]{babel}
\usepackage[utf8x]{inputenc}
\usepackage[T1]{fontenc}

%% Sets page size and margins
\usepackage[a4paper,top=3cm,bottom=2cm,left=3cm,right=3cm,marginparwidth=1.75cm]{geometry}

%% Useful packages
\usepackage{amsmath}
\usepackage{amsfonts}
\usepackage{graphicx}
\usepackage[colorinlistoftodos]{todonotes}
\usepackage[colorlinks=true, allcolors=blue]{hyperref}

\title{Judson's Abstract Algebra: Chapter 6}
\date{}

\begin{document}
\maketitle

\section*{1}

Suppose that $G$ is a finite group with an element $g$ of order 5 and an element $h$ of order 7. Why must $|G| \geq 35$.

\vspace{\baselineskip}

Recall Corollary 6.6: Suppose that $G$ is a finite group and $g \in G$. Then the order of $g$ must divide the number of elements in $G$. Thus $|G|$ must be divisible by 5 and by 7. The smallest such positive integer that has this property is 35. This proves that $|G| \geq 35$.


\section*{2}

Suppose that $G$ is a finite group with 60 elements. What are the orders of possible subgroups of $G$?

\vspace{\baselineskip}

The possible orders of subgroups of $G$ are the divisors of $G$: 1,2,3,4,5,6,10,12,15,20,30,60.


\end{document}