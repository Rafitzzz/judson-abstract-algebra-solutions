\documentclass[a4paper]{article}

%% Language and font encodings
\usepackage[english]{babel}
\usepackage[utf8x]{inputenc}
\usepackage[T1]{fontenc}

%% Sets page size and margins
\usepackage[a4paper,top=3cm,bottom=2cm,left=3cm,right=3cm,marginparwidth=1.75cm]{geometry}

%% Useful packages
\usepackage{amsmath}
\usepackage{amsfonts}
\usepackage{graphicx}
\usepackage[colorinlistoftodos]{todonotes}
\usepackage[colorlinks=true, allcolors=blue]{hyperref}

\title{Judson's Abstract Algebra: Chapter 6}
\date{}

\begin{document}
\maketitle

\section*{1}

Suppose that $G$ is a finite group with an element $g$ of order 5 and an element $h$ of order 7. Why must $|G| \geq 35$.

\vspace{\baselineskip}

Recall Corollary 6.6: Suppose that $G$ is a finite group and $g \in G$. Then the order of $g$ must divide the number of elements in $G$. Thus $|G|$ must be divisible by 5 and by 7. The smallest such positive integer that has this property is 35. This proves that $|G| \geq 35$.


\section*{2}

Suppose that $G$ is a finite group with 60 elements. What are the orders of possible subgroups of $G$?

\vspace{\baselineskip}

The possible orders of subgroups of $G$ are the divisors of $G$: 1,2,3,4,5,6,10,12,15,20,30,60.


\section*{4}

Prove or disprove: Every subgroup of the integers has finite order.

\vspace*{\baselineskip}

This statement is false. Consider the group $2\mathbb{Z} = \{ ..., -2, 0, 2, ... \}$. Note that $0 \in 2\mathbb{Z}$. Note that since the sum of two even numbers is even that $2\mathbb{Z}$ is closed. Notice that if $x \in \mathbb{Z}$ is even then $-x$ is even as well, this proves that all inverses are in $\mathbb{Z}$. This proves that $2\mathbb{Z}$ is a subgroup of $\mathbb{Z}$ with infinite order.


\section*{7}

Verify Euler's Theorem for $n=15$ and $a=4$.

\vspace{\baselineskip}

Recall Euler's Theorem: Let $a$ and $n$ be integers such that $n > 0$ and $\gcd(a,n) = 1$. Then $a^{\phi(n)} \equiv 1 \mod n$. Note that $\phi(15) = 8$. Note that $\gcd(15, 4) = 1$. Consider

$$4^8 = 65536 \equiv 1 \mod 15.$$


\section*{14}

Suppose that $g^n = e$. Show that the order of $g$ divides $n$.

\vspace{\baselineskip}

Clearly $|g| \leq n$. If $|g| = n$, then clearly $|g|$ divides $n$. In the case where $|g| < n$ we will proceed by contradiction. Assume $|g|$ does not divide $n$. Applying the division algorithm yields $n = q|g| + r$ where $0 < r < |g|$. Consider

$$e = g^n = g^{q|g| + r} = g^{q|g|}g^r = eg^r = g^r$$

this is a a contradiction of $0 < r < |g|$, hence $|g|$ divides $n$.




\end{document}