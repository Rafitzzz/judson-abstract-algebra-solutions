\documentclass[a4paper]{article}

%% Language and font encodings
\usepackage[english]{babel}
\usepackage[utf8x]{inputenc}
\usepackage[T1]{fontenc}

%% Sets page size and margins
\usepackage[a4paper,top=3cm,bottom=2cm,left=3cm,right=3cm,marginparwidth=1.75cm]{geometry}

%% Useful packages
\usepackage{amsmath}
\usepackage{amsfonts}
\usepackage{graphicx}
\usepackage[colorinlistoftodos]{todonotes}
\usepackage[colorlinks=true, allcolors=blue]{hyperref}

\title{Judson's Abstract Algebra: Chapter 1}
\date{}

\begin{document}
\maketitle

\section*{1}

Suppose that 

\begin{align*}
A &= \{ x : x \in \mathbb{N} \text{ and } x \text{ is even}\}, \\
B &= \{ x : x \in \mathbb{N} \text{ and } x \text{ is prime}\}, \\
C &= \{ x : x \in \mathbb{N} \text{ and } x \text{ is a multiple of 5}\}.
\end{align*}

Then

\begin{align*}
A \cap B &= \{ 2 \} \\
B \cap C &= \{ 5 \} \\
A \cup B &= \{ x : x \in \mathbb{N} \text{ and } x \text{ is even or } x \text{ is prime}\} \\
A \cap (B \cup C) &= \{ x : x \in \mathbb{N} \text{ and } x \text{ is a multiple of 10}\}
\end{align*}


\section*{2}

If $A = \{ a, b, c \}, B = \{ 1, 2, 3 \}, C = \{ x \}, D = \emptyset$ then

\begin{align*}
A \times B &= \{ (a,1), (a,2), (a,3), (b,1), (b,2), (b,3), (c,1), (c,2), (c,3)\} \\
B \times C &= \{ (1,a), (1,b), (1,c), (2,a), (2,b), (2,c), (3,a), (3,b), (3,c)\} \\
A \times B \times C &= \{ (a,1,x), (a,2,x), (a,3,x), (b,1,x), (b,2,x), (b,3,x), (c,1,x), (c,2,x), (c,3,x)\} \\
A \times D &= \emptyset.
\end{align*}


\section*{3}

Find an example of two nonempty sets $A$ and $B$ for which $A \times B = B \times A$.

Consider any nonempty set $A = B$.


\section*{4}

Prove $A \cup \emptyset = A$.

\vspace{\baselineskip}

By definition $A \cup \emptyset = \{ x : x \in A \lor x \in \emptyset \}$. Note that the second condition is always false and hence

$$A \cup \emptyset = \{ x : x \in A \lor x \in \emptyset \} = \{ x : x \in A \} = A.$$

Prove $A \cap \emptyset = \emptyset$.

\vspace{\baselineskip}

This is very similar to the previous proof. By definition $A \cap \emptyset = \{ x : x \in A \land x \in \emptyset \}$. Note that the second condition is always false and hence

$$A \cap \emptyset = \{ x : x \in A \land x \in \emptyset \} = \{ \} = \emptyset.$$


\section*{5}

Prove $A \cup B = B \cup A$.

\vspace{\baselineskip}

This follows directly from the definition

$$A \cup B = \{x : x \in A \lor x \in B\} = B \cup A.$$

Prove $A \cap B = B \cap A$.

\vspace{\baselineskip}

This follows directly from the definition

$$A \cap B = \{x : x \in A \land x \in B\} = B \cap A.$$


\section*{6}

Prove $A \cup (B \cap C) = (A \cup B) \cap (A \cup C)$.

\vspace{\baselineskip}

Let $x \in A \cup (B \cap C)$ then $x \in A$ or $x \in B \cap C$. If $x \in A$ then $x \in (A \cup B) \cap (A \cup C)$. If $x \in B \cap C$ then $x \in (A \cup B) \cap (A \cup C)$. Hence $A \cup (B \cap C) \subset (A \cup B) \cap (A \cup C)$. 

\vspace{\baselineskip}

Let $x \in (A \cup B) \cap (A \cup C)$. Then $x \in A \cup B$ and $x \in A \cup C$. If $x \in A$, then clearly $x \in A \cup (B \cap C)$. If $x \not\in A$ then $x \in B$ and $x \in C$ hence $x \in A \cap B$. From these two facts we have  $(A \cup B) \cap (A \cup C) \subset A \cup (B \cap C)$.


\section*{7}

Prove $A \cap (B \cup C) = (A \cap B) \cup (A \cap C)$.

\vspace{\baselineskip}

Let $x \in A \cap (B \cup C)$. Then $x\in A$ and $x \in B \cup C$. There are two cases. If $x \in B$ then $x \in A \cap B$ by definition. Similarly, if $x \in C$ then $x \in A \cap C$. Hence, by definition, $x \in (A \cap B) \cup (A \cap C)$. This shows $A \cap (B \cup C) \subset (A \cap B) \cup (A \cap C)$.

\vspace{\baselineskip}

Let $x \in (A \cap B) \cup (A \cap C)$. There are two cases. If $x \in A \cap B$ then clearly $x \in A \cap (B \cup C)$. Similarly, if $x \in A \cap C$ then clearly $x \in A \cap (B \cup C)$. This shows $(A \cap B) \cup (A \cap C) \subset A \cap (B \cup C)$. This proves $A \cap (B \cup C) = (A \cap B) \cup (A \cap C)$.


\section*{8}

Prove $A \subset B$ if and only if $A \cap B = A$.

\vspace{\baselineskip}

Assume $A \cap B = A$. Then  for all $a \in A$ it is true that $a \in A \cap B$ and thus $a \in B$. Hence $A \subset B$.

\vspace{\baselineskip}

Assume $A \subset B$. Then for all $a \in A$ it is true that $a \in B$. Since $a \in A$ and $a \in B$ we know $A \subset A \cap B$. Clearly $A \cap B \subset A$, therefore $A \cap B = A$.


\section*{9}

Prove $(A \cap B)' = A' \cup B'$.

\vspace{\baselineskip}

Let $x \in (A \cap B)'$. Then $x \not \in (A \cap B)$. By the definition of intersection either $x \not \in A$ or $x \not \in B$ so $x \in A'$ or $x \in B'$. By definition of union $x \in A' \cup B'$. This shows $(A \cap B)' \subseteq A' \cup B'$.

\vspace{\baselineskip}

Let $x \in A' \cup B'$. Then $x \in A'$ or $x \in B'$ by the definition of union. By the definition of the complement $x \not \in A$ or $x \not \in B$ which implies that $x \not \in A \cap B$. Then, by the definition of the complement, $x \in (A \cap B)'$. This shows $A' \cup B' \subseteq (A \cap B)'$.  Hence $(A \cap B)' = A' \cup B'$.



\section*{10}


\section*{11}


\section*{12}

Prove $(A \cap B) \setminus B = \emptyset$.

\vspace{\baselineskip}

Let $ x \in (A \cap B) \setminus B$. Then $x \in A \cap B$ and $x \not\in B$. This is a contradiction since $x \in A \cap B$ implies $x \in B$. Hence, there are no such $x$. This is what we wanted to prove.


\section*{13}


\section*{14}


\section*{15}


\section*{16}


\section*{17}


\section*{18}

Determine which of the following functions are one-to-one and which are onto. If the function is not onto, determine it's range.

\vspace{\baselineskip}

$f : \mathbb{R} \rightarrow \mathbb{R}$ defined by $f(x) = e^x$ is one-to-one (note that it is a continuous, strictly increasing function), but not onto. It's range is the positive real numbers.

\vspace{\baselineskip}

$f : \mathbb{Z} \rightarrow \mathbb{Z}$ defined by $f(n) = n^2 + 3$ is not one-to-one or onto. Note that $f(-1) = 4 = f(1)$ and that there is no solution to $f(n) = 1$. The range of $f$ is $\{ n^ + 3, n \in \mathbb{N} \}$.

\vspace{\baselineskip}

$f : \mathbb{R} \rightarrow \mathbb{R}$ defined by $f(x) = \sin(x)$ is not one-to-one or onto. Note that $\sin(0) = \sin(2\pi)$. The range of $f$ is $[-1, 1]$.

\vspace{\baselineskip}

$f : \mathbb{Z} \rightarrow \mathbb{Z}$ defined by $f(x) = x^2$ is not one-to-one or onto. Note that $f(-1) = 1 = f(1)$ and that there is no solution to $f(x) = -1$. The range of $f$ is $\{ x^2, x \in \mathbb{N}\}$.


\section*{19}

Let $f: A \rightarrow B$ and $g: B \rightarrow C$ be invertible mappings; that is, mappings such that $f^{-1}$ and $g^{-1}$ exist. Show that $(g \circ f)^{-1} = f^{-1} \circ g^{-1}$.

\vspace{\baselineskip}

Let $c \in C$. Note that there exists $a \in A$ such that 

$$(g \circ f)^{-1}(c) = a$$

and

$$(g \circ f)(a) = c.$$

Consider the second equation

\begin{align*}
(g \circ f)(a) &= c \\ 
g(f(a)) &= c \\
f^{-1}(g^{-1}(g(f(a)))) &= f^{-1}(g^{-1}(c)) \\
a &= f^{-1}(g^{-1}(c)) \\
a &= (f^{-1} \circ g^{-1}) (c).
\end{align*}

Combining this result and the fact that $(g \circ f)^{-1}(c) = a$ yields

$$(g \circ f)^{-1}(c) = a = (f^{-1} \circ g^{-1})(c).$$

This is what we are trying to prove.


\section*{20}

Define a function $f: \mathbb{N} \rightarrow \mathbb{N}$ that is one-to-one but not onto.

\vspace{\baselineskip}

Let $m,n \in \mathbb{N}$. Consider the function $f(x) = 2x$. We know $f$ is one-to-one since

\begin{align*}
f(m) &= f(n) \\
2m &= 2n \\
m &= n
\end{align*}

but notice that $f$ is not onto since there is no solution to f(x) = 3 (or any other odd number).

\vspace{\baselineskip}

Define a function $f: \mathbb{N} \rightarrow \mathbb{N}$ that is onto but not one-to-one.

\vspace{\baselineskip}

Let $m,n \in \mathbb{N}$. Consider the function $f(x) = x - 1$ if $n > 1$ and $f(1) = 1$. Notice that $f$ is onto since $f(n+1) = n$ where $n$ is arbitrary. A counterexample to $f$ being one-to-one is $f(1) = f(2) = 1$.

\end{document}