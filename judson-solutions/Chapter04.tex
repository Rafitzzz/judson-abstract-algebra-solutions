\documentclass[a4paper]{article}

%% Language and font encodings
\usepackage[english]{babel}
\usepackage[utf8x]{inputenc}
\usepackage[T1]{fontenc}

%% Sets page size and margins
\usepackage[a4paper,top=3cm,bottom=2cm,left=3cm,right=3cm,marginparwidth=1.75cm]{geometry}

%% Useful packages
\usepackage{amsmath}
\usepackage{amsfonts}
\usepackage{listings}
\usepackage{graphicx}
\usepackage[colorinlistoftodos]{todonotes}
\usepackage[colorlinks=true, allcolors=blue]{hyperref}

\title{Judson's Abstract Algebra: Chapter 4}
\date{}

\begin{document}
\maketitle

\section*{1}

\section*{2}

\section*{3}

\section*{4}


\section*{5}

Find the order of every element in $\mathbb{Z}_{18}$.

\vspace{\baselineskip}

\begin{tabular}{ l | c }
  Element & Order \\
  \hline
  0 & 1 \\
  1 & 18 \\
  2 & 9 \\
  3 & 6 \\
  4 & 9 \\
  5 & 18 \\
  6 & 3 \\
  7 & 18 \\
  8 & 9 \\
  9 & 2 \\
  10 & 9 \\
  11 & 18 \\
  12 & 3 \\
  13 & 18 \\
  14 & 9 \\
  15 & 6 \\
  16 & 9 \\
  17 & 18 \\
\end{tabular}

This can be generated using a simple python script

\begin{verbatim}
for i in range(18):
    for j in range(1, 19):
        if (i*j) % 18 == 0:
            print i,j
            break
\end{verbatim}


\section*{6}


\section*{7}


\section*{8}


\section*{9}


\section*{10}

Find all elements of finite order in each of the following groups. Here the "*" indicates the set with 0 removed.

\vspace{\baselineskip}

In $\mathbb{Z}$ the single element with finite order is $0.$

In $\mathbb{Q}^*$ the elements with finite order are $1, -1.$

In $\mathbb{R}^*$ the elements with finite order are $1, -1.$


\section*{11}

If $a^{24} = e$ in a group $G$, what are possible orders of $a$?

\vspace{\baselineskip}

Possible orders of $a$ are factors of 24. Therefore $|a| \in \{ 1, 2, 3, 4, 6, 8, 12, 24 \}$.


\section*{12}


\section*{13}


\section*{14}


\section*{15}

Evaluate each of the following

$$(3-2i) + (5i-6) = -3 + 3i$$
$$(4-5i) - \overline{(4i-4)} = 8 - i$$
$$(5-4i)(7+2i) = 35 + 10i - 28i + 8 = 43 - 18i$$
$$(9-i)\overline{(9-i)} = 9^2 + 1^2 = 82$$
$$i^{45} = i^{44} i = i(i^4)^{11} = i 1^{11} = i$$
$$(1+i) + \overline{(1+i)} = 2$$


\section*{16}

Convert the following complex numbers to the form $a + bi$. 

$$2 \text{cis} (\pi/6) = 2(\cos (\pi/6) + i \sin(\pi/6)) = \sqrt{3} + i$$
$$5 \text{cis} (9\pi/4) = 5(\cos (9\pi/4) + i \sin(9\pi/4)) = \frac{5 + 5i}{2}$$
$$4 \text{cis} (\pi) = 4(\cos (\pi) + i \sin(\pi)) = -4$$
$$\text{cis} (7\pi/4) = \cos (7\pi/4) + i \sin(7\pi/4) = \frac{1-i}{/sqrt{2}}$$


\section*{17}

Convert the following complex numbers to polar representation.

$$ 1-i = \sqrt{2} \text{cis}(\pi/4)$$
$$ -5 = 5(\text{cis}(\pi)$$
$$ 2 + 2i = 2\sqrt{2}(\text{cis}(\pi/4)$$
$$ \sqrt{3} + i = 2(\text{cis}(\pi/6)$$
$$ -3i = 3(\text{cis}(-\pi/2)$$
$$ 2i + 2 \sqrt{3} = 4(\text{cis}(\pi/6)$$


\section*{18}




\section*{31}

Let $G$ be an abelian group. Show that the elements of finite order in $G$ form a subgroup. This subgroup is called the \textit{torsion subgroup} of $G$.

\vspace{\baselineskip}

Denote the torsion subgroup of $G$ by $T$. Note that $e$ has order 1 and hence $e \in T$. Let $t \in T$. Let $n$ be the order of $t$ such that $t^n = e$. Notice that since $tt^{n-1} = e$ it is true that $t^{-1} = t^{n-1}$ and thus $t^{-1} \in T$. Let $y \in T$ with the order of $y$ being $m$. Notice that $ty^{mn} = t^{mn}y^{mn} = e$. This shows $ty \in T$. This proves that $T$ is a subgroup of $G$. 


\section*{41}

Prove that the circle group is a subgroup of $\mathbb{C}^*$.

\vspace{\baselineskip}

Note that since $|1| = 1$ the identity is in $\mathbb{T}$. Let $x,y \in \mathbb{T}$. Notice that $|xy| = |x||y| = 1$. This implies that $\mathbb{T}$ is closed under the group operation. Consider 

\begin{align*}
x x^{-1} &= 1 \\
|x x ^{-1}| &= |1| \\
|x x ^{-1}| &= 1 \\
|x| |x ^{-1}| &= 1 \\
1 | x ^{-1}| &= 1 \\
|x ^{-1}| &= 1 .\\
\end{align*}

This shows $x^{-1} \in \mathbb{T}$. This shows the circle group is a subgroup of $\mathbb{C}^*$.


















\end{document}