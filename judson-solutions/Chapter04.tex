\documentclass[a4paper]{article}

%% Language and font encodings
\usepackage[english]{babel}
\usepackage[utf8x]{inputenc}
\usepackage[T1]{fontenc}

%% Sets page size and margins
\usepackage[a4paper,top=3cm,bottom=2cm,left=3cm,right=3cm,marginparwidth=1.75cm]{geometry}

%% Useful packages
\usepackage{amsmath}
\usepackage{amsfonts}
\usepackage{listings}
\usepackage{graphicx}
\usepackage[colorinlistoftodos]{todonotes}
\usepackage[colorlinks=true, allcolors=blue]{hyperref}

\title{Judson's Abstract Algebra: Chapter 4}
\date{}

\begin{document}
\maketitle

\section{}

\section{}

\section{}

\section{}


\section{}

Find the order of every element in $\mathbb{Z}_{18}$.

\vspace{\baselineskip}

\begin{tabular}{ l | c }
  Element & Order \\
  \hline
  0 & 1 \\
  1 & 18 \\
  2 & 9 \\
  3 & 6 \\
  4 & 9 \\
  5 & 18 \\
  6 & 3 \\
  7 & 18 \\
  8 & 9 \\
  9 & 2 \\
  10 & 9 \\
  11 & 18 \\
  12 & 3 \\
  13 & 18 \\
  14 & 9 \\
  15 & 6 \\
  16 & 9 \\
  17 & 18 \\
\end{tabular}

This can be generated using a simple python script

\begin{verbatim}
for i in range(18):
    for j in range(1, 19):
        if (i*j) % 18 == 0:
            print i,j
            break
\end{verbatim}


\section{}


\section{}


\section{}


\section{}


\section{}

If $a^{24} = e$ in a group $G$, what are possible orders of $a$?

\vspace{\baselineskip}

Possible orders of $a$ are factors of 24. Therefore $|a| \in \{ 1, 2, 3, 4, 6, 8, 12, 24 \}$.


\section{}


\section{}


\section{}


\section{}


\section{}


\section{}

Convert the following complex numbers to the form $a + bi$. 

$$2 \text{cis} (\pi/6) = 2(\cos (\pi/6) + i \sin(\pi/6)) = \sqrt{3} + i$$
$$5 \text{cis} (9\pi/4) = 5(\cos (9\pi/4) + i \sin(9\pi/4)) = \frac{5 + 5i}{2}$$
$$4 \text{cis} (\pi) = 4(\cos (\pi) + i \sin(\pi)) = -4$$
$$\text{cis} (7\pi/4) = \cos (7\pi/4) + i \sin(7\pi/4) = \frac{1-i}{/sqrt{2}}$$


\section{}

Convert the following complex numbers to polar representation.

$$ 1-i = \sqrt{2} \text{cis}(\pi/4)$$
$$ -5 = 5(\text{cis}(\pi)$$
$$ 2 + 2i = 2\sqrt{2}(\text{cis}(\pi/4)$$
$$ \sqrt{3} + i = 2(\text{cis}(\pi/6)$$
$$ -3i = 3(\text{cis}(-\pi/2)$$
$$ 2i + 2 \sqrt{3} = 4(\text{cis}(\pi/6)$$
























\end{document}