\documentclass[a4paper]{article}

%% Language and font encodings
\usepackage[english]{babel}
\usepackage[utf8x]{inputenc}
\usepackage[T1]{fontenc}

%% Sets page size and margins
\usepackage[a4paper,top=3cm,bottom=2cm,left=3cm,right=3cm,marginparwidth=1.75cm]{geometry}

%% Useful packages
\usepackage{amsmath}
\usepackage{amsfonts}
\usepackage{listings}
\usepackage{graphicx}
\usepackage[colorinlistoftodos]{todonotes}
\usepackage[colorlinks=true, allcolors=blue]{hyperref}

\title{Judson's Abstract Algebra: Chapter 4}
\date{}

\begin{document}
\maketitle

\section*{1}

\section*{2}

\section*{3}

\section*{4}


\section*{5}

Find the order of every element in $\mathbb{Z}_{18}$.

\vspace{\baselineskip}

\begin{tabular}{ l | c }
  Element & Order \\
  \hline
  0 & 1 \\
  1 & 18 \\
  2 & 9 \\
  3 & 6 \\
  4 & 9 \\
  5 & 18 \\
  6 & 3 \\
  7 & 18 \\
  8 & 9 \\
  9 & 2 \\
  10 & 9 \\
  11 & 18 \\
  12 & 3 \\
  13 & 18 \\
  14 & 9 \\
  15 & 6 \\
  16 & 9 \\
  17 & 18 \\
\end{tabular}

This can be generated using a simple python script

\begin{verbatim}
for i in range(18):
    for j in range(1, 19):
        if (i*j) % 18 == 0:
            print i,j
            break
\end{verbatim}


\section*{6}


\section*{7}


\section*{8}


\section*{9}


\section*{10}

Find all elements of finite order in each of the following groups. Here the "*" indicates the set with 0 removed.

\vspace{\baselineskip}

In $\mathbb{Z}$ the single element with finite order is $0.$

In $\mathbb{Q}^*$ the elements with finite order are $1, -1.$

In $\mathbb{R}^*$ the elements with finite order are $1, -1.$


\section*{11}

If $a^{24} = e$ in a group $G$, what are possible orders of $a$?

\vspace{\baselineskip}

Possible orders of $a$ are factors of 24. Therefore $|a| \in \{ 1, 2, 3, 4, 6, 8, 12, 24 \}$.


\section*{12}


\section*{13}


\section*{14}


\section*{15}

Evaluate each of the following

$$(3-2i) + (5i-6) = -3 + 3i$$
$$(4-5i) - \overline{(4i-4)} = 8 - i$$
$$(5-4i)(7+2i) = 35 + 10i - 28i + 8 = 43 - 18i$$
$$(9-i)\overline{(9-i)} = 9^2 + 1^2 = 82$$
$$i^{45} = i^{44} i = i(i^4)^{11} = i 1^{11} = i$$
$$(1+i) + \overline{(1+i)} = 2$$


\section*{16}

Convert the following complex numbers to the form $a + bi$. 

$$2 \text{cis} (\pi/6) = 2(\cos (\pi/6) + i \sin(\pi/6)) = \sqrt{3} + i$$
$$5 \text{cis} (9\pi/4) = 5(\cos (9\pi/4) + i \sin(9\pi/4)) = \frac{5 + 5i}{2}$$
$$4 \text{cis} (\pi) = 4(\cos (\pi) + i \sin(\pi)) = -4$$
$$\text{cis} (7\pi/4) = \cos (7\pi/4) + i \sin(7\pi/4) = \frac{1-i}{/sqrt{2}}$$


\section*{17}

Convert the following complex numbers to polar representation.

$$ 1-i = \sqrt{2} \text{cis}(\pi/4)$$
$$ -5 = 5(\text{cis}(\pi)$$
$$ 2 + 2i = 2\sqrt{2}(\text{cis}(\pi/4)$$
$$ \sqrt{3} + i = 2(\text{cis}(\pi/6)$$
$$ -3i = 3(\text{cis}(-\pi/2)$$
$$ 2i + 2 \sqrt{3} = 4(\text{cis}(\pi/6)$$


\section*{18}

Calculate each of the following expressions

$$(1+i)^{-1} = 1/2 - i/2$$
$$(1-i)^6 = 8i$$
$$(\sqrt{3} + i)^5 = 16 (-\sqrt(3) + i)$$
$$(-i)^{10} = -1$$
$$ ((1-i)/2)^4 = -1/4$$
$$(-\sqrt{2} - \sqrt{2} i)^{12} = -4096$$
$$(-2 + 2i)^{-5} = (1/i)/256$$


\section*{19}

Prove each of the following statements.

$$|z| = |\overline{z}|$$
$$z \overline{z} = |z|^2$$
$$z^{-1} = \overline{z} / |z|^2$$
$$|z+w| \leq |z| + |w|$$
$$|z-w| \geq ||z| - |w||$$
$$|zw| = |z||w| $$

These statements are proved in order, but some proofs may rely on statements yet to be proved. Let $z = a + bi$ and $w = c + di$. Consider

\begin{align*}
|z| &= |a + bi| \\
&= \sqrt{a^2 + b^2} \\
&= \sqrt{a^2 + (-b)^2} \\
&= |a - bi| \\
&= |\overline{z}|.
\end{align*}

This proves the first statement. Consider

\begin{align*}
z \overline{z} &= (a + bi)(a - bi) \\
&= a^2 + b^2 \\
&= |z|^2.
\end{align*}

This proves the second statement. Consider, that by the previous statement,

\begin{align*}
z \overline{z} / |z|^2 &= |z|^2 / |z|^2 \\
&= 1.
\end{align*}

This proves the third statement. To prove the fourth statement first consider that

\begin{align*}
z \overline{w} + w \overline{z} &= z \overline{w} + \overline{z \overline{w}} \\
&= 2 \text{Re} (z \overline{w}) \\
&\leq 2 |z \overline{w}| = |z||\overline{w}| = |z||w|.
\end{align*}

Using this fact

\begin{align*}
|z+w|^2 &= (z+w) \overline{(z+w)} \\
&= (z+w) (\overline{z} + \overline{w}) \\
&= z \overline{z} + z \overline{w} + w \overline{z} + w \overline{w} \\
&= |z|^2 + z \overline{w} + w \overline{z} + |w|^2 \\
&\leq |z|^2 + 2|z||w| + |w|^2 = (|z| + |w|)^2.
\end{align*}

Since the modulus is a greater than or equal to 0 we can take the square root of both sides. This proves the fourth statement. Assume that $|z| \geq |w|$. Consider

\begin{align*}
|z| &= |(z-w) - w| \\
 &\leq |z-w| + |-w| \\
 & \leq |z-w| + |w|
\end{align*}

Rearranging yields

$$ |z| - |w| \leq |z-w|$$

and since $|z| > |w|$ 

$$ ||z| - |w|| \leq |z-w|.$$

The case where $|w| \geq |z|$ is identical, but uses the fact that

$$|w - z| = |-(z-w)| = |z-w|.$$

This proves the fifth statement. Consider

\begin{align*}
|zw|^2 &= (zw)\overline{(zw)} \\
&= zw \overline{z} \overline{w} \\
&= |z|^2 |w|^2
\end{align*}

Since the modulus is a greater than or equal to 0 we can take the square root of both sides. This proves the sixth statement.




\section*{23}

Let $a,b \in G$. Prove the following statements

The order of $a$ is the same as the order of $a^{-1}$.

For all $g \in G, |a| = |g^{-1} a g|.$

The order of $ab$ is the same as the order of $ba$.

\vspace{\baselineskip}

Let $n$ be the order of $a$. Consider

$$a^n a^{-n} = e$$
$$e a^{-n} = e$$
$$(a^{-1})^{n} = e.$$

We will prove the first statement. We must now show that $n$ is the least positive integer such that this is true for $a^{-n}$. Assume that $m \in \mathbb{Z} : 0 < m < n$ and  $(a^{-1})^{m} = e$. Consider

\begin{align*}
a^m a^{-m} &= e \\
a^m e &= e \\
a^m &= e \\
\end{align*}

which is clearly a contradiction. Hence $|a^{-1}| = n$.

\vspace{\baselineskip}

We will prove the second statement. The proof is similar to that of the previous statement. Let $g \in G$. Consider 

\begin{align*}
(g^{-1} a g)^n &= g^{-1} a g g^{-1} a ... a g g^{-1} a g \\
&= g^{-1} a^n g \\
&= g^{-1} e g \\
&= g^{-1} g \\
&= e
\end{align*} 

We must now show that $n$ is the least positive integer such that this is true for $a^{-n}$. Assume that $m \in \mathbb{Z} : 0 < m < n$ and  $(g^{-1} a g)^{m} = e$. Consider

\begin{align*}
(g^{-1} a g)^{m} &= e \\
g^{-1} a^m g &= e \\
a^m g &= g \\ 
a^m &= gg^{-1} \\ 
a^m &= e
\end{align*}

which is clearly a contradiction.

\vspace{\baselineskip}

We will prove the third statement. The proof is similar to the other two. Let $p$ be the order of $ab$. Consider

\begin{align*}
(ba)^{p+1} &= b (ab)^p a \\
&= b e a \\
&= ba \\
\end{align*}

hence $(ba)^p = e$. We must now show that $p$ is the least positive integer such that this is true for $ba^{p}$. Assume that $q \in \mathbb{Z} : 0 < q < p$ and  $(ba)^{q} = e$. Consider

\begin{align*}
(ab)^{q+1} &= a (ba)^q b \\
&= a e b \\
&= ab
\end{align*}

hence $(ab)^q = e$ which is a contradiction. 



\section*{24}

Let $p$ and $q$ be distinct primes. How many generators does $\mathbb{Z}_{pq}$ have?

\vspace{\baselineskip}

Recall Corollary 4.7: The generators of $\mathbb{Z}_n$ are the integers $r$ such that $1 \leq r < n$ and $\gcd(r,n) = 1$. From this and the fact that $p$ and $q$ are the prime factors of $pq$ we know that each $r$ that is not a multiple of $p$ and not a multiple of $q$ is a generator. There are $q$ multiples of $p$ between 0 and $pq$ and there are $p$ multiples of $q$ between 0 and $pq$. This double counts 0 so we must subtract it. Therefore, there are $pq - p - q + 1$ generators in $\mathbb{Z}_n$.


\section*{25}

Let $p$ be prime and $r$ be a positive integer. How many generators does $\mathbb{Z}_{p^r}$ have?

\vspace{\baselineskip}

Recall Corollary 4.7: The generators of $\mathbb{Z}_n$ are the integers $r$ such that $1 \leq r < n$ and $\gcd(r,n) = 1$. The proof is similar to that of exercise 24. The generators are those elements of $\mathbb{Z}_{p^r}$ that are not multiples of $p$. There are $p^r / p = p^{r-1}$ multiples of $p$ in $\mathbb{Z}_{p^r}$. Therefore $\mathbb{Z}_{p^r}$ has $p^r - p^{r-1} = (p-1)p^{r-1}$ generators.


\section*{26}

Prove that $\mathbb{Z}_p$ has no nontrivial subgroups if $p$ is prime.

\vspace{\baselineskip}

Recall Corollary 4.7: The generators of $\mathbb{Z}_n$ are the integers $r$ such that $1 \leq r < n$ and $\gcd(r,n) = 1$. Note that except for 0 all elements  $x \in \mathbb{Z}_p$ have $\gcd(x, p) = 1$. That means for all subgroups except for $\{ 0 \}$ that if the subgroup contains a non-zero element the subgroup contains the entire group. This shows that there are no nontrivial subgroups of $\mathbb{Z}_p$ if $p$ is prime.




\section*{31}


Let $G$ be an abelian group. Show that the elements of finite order in $G$ form a subgroup. This subgroup is called the \textit{torsion subgroup} of $G$.

\vspace{\baselineskip}

Denote the torsion subgroup of $G$ by $T$. Note that $e$ has order 1 and hence $e \in T$. Let $t \in T$. Let $n$ be the order of $t$ such that $t^n = e$. Notice that since $tt^{n-1} = e$ it is true that $t^{-1} = t^{n-1}$ and thus $t^{-1} \in T$. Let $y \in T$ with the order of $y$ being $m$. Notice that $(ty)^{mn} = t^{mn}y^{mn} = e$. This shows $ty \in T$. This proves that $T$ is a subgroup of $G$. 


\section*{38}

Prove that the order of an element in a cyclic group $G$ must divide the order of the group.

\vspace{\baselineskip}

Read prop 4.5 for idea of proof


\section*{41}

Prove that the circle group is a subgroup of $\mathbb{C}^*$.

\vspace{\baselineskip}

Note that since $|1| = 1$ the identity is in $\mathbb{T}$. Let $x,y \in \mathbb{T}$. Notice that $|xy| = |x||y| = 1$. This implies that $\mathbb{T}$ is closed under the group operation. Consider 

\begin{align*}
x x^{-1} &= 1 \\
|x x ^{-1}| &= |1| \\
|x x ^{-1}| &= 1 \\
|x| |x ^{-1}| &= 1 \\
1 | x ^{-1}| &= 1 \\
|x ^{-1}| &= 1 .\\
\end{align*}

This shows $x^{-1} \in \mathbb{T}$. This shows the circle group is a subgroup of $\mathbb{C}^*$.


















\end{document}